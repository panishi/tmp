\documentclass[11pt, a4paper]{jsarticle}

\usepackage{amsmath, amssymb, amsthm}

\newcommand{\E}{\mathbb{E}}
\newcommand{\FP}{\textbf{FP}}
\newcommand{\PS}{\textbf{PS}}
\newcommand{\BC}{\textbf{BC}}
\newcommand{\BP}{\textbf{BP}}

\newtheorem{assumption}{Assumption}[section]
\newtheorem{proposition}[assumption]{Proposition}
\newtheorem{theorem}[assumption]{Theorem}
\newtheorem{exercise}{Exercise}[section]

\begin{document}

\setcounter{section}{1}
\section{Vanilla interest rate options and forward measure}
\begin{assumption}
リスク中立測度$Q$が存在して,価格$V$が
\begin{align}
V(t)
& =
B(t) \E_Q \Bigl[ \frac{V(T)}{B(T)} \Bigl| \mathcal{F}_t \Bigr] \nonumber
\end{align}
と表されることを仮定する.
\end{assumption}

\subsection{Change of numeraire}
$A$をニューメレールとする.

\begin{proposition}
\begin{align}
V(t)
& =
A(t) \E_{P_A} \Bigl[ \frac{V(T)}{A(T)} \Bigl| \mathcal{F}_t \Bigr] \nonumber
\end{align}
\end{proposition}

\begin{proof}
$A(t) / B(t)$は$Q$-マルチンゲールなので
\begin{align}
\E_Q \Bigl[ \frac{A(T)}{B(T)} \Bigr]
& =
\frac{A(0)}{B(0)} = A(0) \nonumber
\end{align}

よって
\begin{align}
\frac{d P_A}{d Q}
& =
\frac{1}{A(0)} \frac{A(T)}{B(T)} \nonumber
\end{align}
によって確率測度$P_A$を定められる.

マルチンゲール性より
\begin{align}
\frac{d P_A}{d Q} \Bigr|_t
& =
\E_Q \Bigl[ \frac{d P_A}{d Q} \Big| \mathcal{F}_t \Bigr]
= \frac{1}{A(0)} \frac{A(t)}{B(t)} \nonumber
\end{align}

条件付期待値についてのベイズの定理より
\begin{align}
\E_Q \Bigl[ X \frac{d P_A}{d Q} \Bigl| \mathcal{F}_t \Bigr]
& =
\frac{d P_A}{d Q} \Bigr|_t \E_{P_A} [X | \mathcal{F}_t] \nonumber
\end{align}

ゆえに
\begin{align}
V(t)
& =
B(t) \E_Q \Bigl[ \frac{V(T)}{B(T)} \Bigl| \mathcal{F}_t \Bigr] \nonumber \\
& =
A(0) B(t) \E_Q \Bigl[ \frac{V(T)}{A(T)} \frac{d P_A}{d Q} \Bigl| \mathcal{F}_t \Bigr] \nonumber \\
& =
A(0) B(t) \frac{d P_A}{d Q} \Bigr|_t \E_{P_A} \Bigl[ \frac{V(T)}{A(T)} \Bigl| \mathcal{F}_t \Bigr] \nonumber \\
& =
A(t) \E_{P_A} \Bigl[ \frac{V(T)}{A(T)} \Bigl| \mathcal{F}_t \Bigr] \nonumber
\end{align}
\end{proof}

\hrulefill
\begin{exercise}
$t=0$とおくと
\begin{align}
V(0)
& =
A(0) \E_{P_A} \Bigl[ \frac{X}{A(T)} \Bigr] \nonumber
\end{align}
\end{exercise}
\hrulefill \\

\subsection{Forward measure}
ニューメレールは$B(t, T)$.
\begin{align}
\frac{d P_T}{d Q}
& =
\frac{1}{B(0, T)} \frac{B(T, T)}{B(T)}
= \frac{1}{B(0, T) B(T)} \nonumber
\end{align}

\begin{align}
\frac{d P_T}{d Q} \Bigr|_t
& =
\frac{1}{B(0, T)} \frac{B(t, T)}{B(t)}
= \frac{B(t, T)}{B(0, T) B(t)} \nonumber
\end{align}

\hrulefill
\begin{exercise}
$S \leq T$のとき
\begin{align}
\frac{dP_S}{dP_T}
& =
\frac{dP_S}{dQ} \Bigl/ \frac{dP_T}{dQ} \Bigr|_S
= \frac{B(0, T)}{B(0, S)} \frac{1}{B(S, T)} \nonumber
\end{align}

$S \geq T$のとき
\begin{align}
\frac{dP_S}{dP_T}
& =
\frac{dP_S}{dQ} \Bigr|_T \Bigl/ \frac{dP_T}{dQ}
= \frac{B(0, T)}{B(0, S)} B(S, T) \nonumber
\end{align}
\end{exercise}
\hrulefill \\

$T$時点のペイオフ$X$に対して
\begin{align}
B(t) \E_Q \Bigl[ \frac{X}{B(T)} \Bigl| \mathcal{F}_t \Bigr]
& =
B(t, T) \E_{P_T} \Bigl[ \frac{X}{B(T, T)} \Bigl| \mathcal{F}_t \Bigr]
= B(t, T) \E_{P_T} [ X | \mathcal{F}_t ] \nonumber
\nonumber
\end{align}

上記の一般化として,$S \leq T$と$S$時点のペイオフ$X$に対して
\begin{align}
B(t) \E_Q \Bigl[ \frac{B(S, T)}{B(S)} X \Bigl| \mathcal{F}_t \Bigr]
& =
B(0, T) B(t) \E_Q \Bigl[ X \frac{d P_T}{d Q} \Bigr|_S \Bigl| \mathcal{F}_t \Bigr] \nonumber \\
& =
B(0, T) B(t) \E_Q \Bigl[ X \frac{d P_T}{d Q} \Bigl| \mathcal{F}_t \Bigr] \nonumber \\
& =
B(0, T) B(t) \frac{d P_T}{d Q} \Bigr|_t \E_{P_T} [ X | \mathcal{F}_t ] \nonumber \\
& =
B(t, T) \E_{P_T} [ X | \mathcal{F}_t ] \nonumber
\nonumber
\end{align}

以下では,$B(t, T)$が以下のSDEに従うと仮定する.
\begin{align}
\frac{d B(t, T)}{B(t, T)}
& =
r(t) dt + \Sigma(t, T) dW(t), \nonumber \\
B(T, T)
& =
1 \nonumber
\end{align}

\hrulefill
\begin{exercise}
$B(t, T) / B(t)$はマルチンゲールなので,ドリフトが$\mathrm{0}$である.
そのためこの伊藤微分を計算したい.
伊藤の商の公式を使っても良いが,計算を楽にするために対数を取る.
\begin{align}
d \log \frac{B(t, T)}{B(t)}
& =
\Bigl( \mu(t, T) - \frac{1}{2} \Sigma(t, T)^2 - r(t) \Bigr) dt + \Sigma(t, T) dW(t) \nonumber
\end{align}

これより
\begin{align}
d \frac{B(t, T)}{B(t)}
& =
\frac{B(t, T)}{B(t)} \Bigl\{ \Bigl( \mu(t, T) - r(t) \Bigr) dt + \Sigma(t, T) dW(t) \Bigr\} \nonumber
\end{align}

よって,$\mu(t, T) = r(t)$である.
\end{exercise}
\hrulefill \\

\hrulefill
\begin{exercise}
\begin{align}
B(t, T)
& =
B(0, T) \exp \Bigl( \int_0^t \Bigl( r(u) - \frac{1}{2} \Sigma(u, T)^2 \Bigr) du + \int_0^t \Sigma(u, T) dW(u) \Bigr) \nonumber
\end{align}

$B(T, T) = 1$より
\begin{align}
B(0, T)
& =
\exp \Bigl( -\int_0^T \Bigl( r(u) - \frac{1}{2} \Sigma(u, T)^2 \Bigr) du - \int_0^T \Sigma(u, T) dW(u) \Bigr) \nonumber
\end{align}

よって
\begin{align}
B(t, T)
& =
\exp \Bigl( -\int_t^T \Bigl( r(u) - \frac{1}{2} \Sigma(u, T)^2 \Bigr) du - \int_t^T \Sigma(u, T) dW(u) \Bigr) \nonumber
\end{align}
\end{exercise}
\hrulefill \\

\hrulefill
\begin{exercise}
\begin{align}
d \frac{B(t, T)}{B(t)}
& =
\frac{B(t, T)}{B(t)} \Sigma(t, T) dW(t) \nonumber
\end{align}
より
\begin{align}
d \xi(t)
& =
\xi(t) \Sigma(t, T) dW(t) \nonumber
\end{align}
\end{exercise}
\hrulefill \\

$\xi(0) = 1$より
\begin{align}
\xi(t)
& =
\exp \Bigl( \int_0^t \Sigma(u, T) dW(u) - \frac{1}{2} \int_0^t \Sigma(u, T)^2 du \Bigr) \nonumber
\end{align}

\begin{proposition}
\begin{align}
\frac{dP_T}{dQ}
& =
\exp \Bigl( \int_0^T \Sigma(u, T) dW(u) - \frac{1}{2} \int_0^T \Sigma(u, T)^2 du \Bigr) \nonumber
\end{align}
\end{proposition}

\hrulefill
\begin{exercise}
$\xi$は
\begin{align}
d \xi(t)
& =
\xi(t) \sum_{i=1}^n \Sigma_i(t, T) dW_i(t) \nonumber
\end{align}
を満たす.

よって
\begin{align}
\xi(t)
& =
\exp \Bigl( \sum_{i=1}^n \int_0^t \Sigma_i(u, T) dW_i(u) - \frac{1}{2} \sum_{i=1}^n \int_0^t \Sigma_i(u, T)^2 du \Bigr), \nonumber \\
\frac{dP_T}{dQ}
& =
\exp \Bigl( \sum_{i=1}^n \int_0^T \Sigma_i(u, T) dW_i(u) - \frac{1}{2} \sum_{i=1}^n \int_0^T \Sigma_i(u, T)^2 du \Bigr) \nonumber
\end{align}
\end{exercise}
\hrulefill \\

\subsection{Forward contract}
\begin{align}
V(t)
& =
B(t, T) \E_{P_T} [ (X(T) - F(t, T)) | \mathcal{F}_t ]
= 0 \nonumber
\end{align}

\begin{align}
F(t, T)
& =
\E_{P_T} [ X(T) | \mathcal{F}_t ] \nonumber
\end{align}

\subsection{Martingales under the forward measure}
$\FP(t; S, T)$は$B(S, T)$のフォワードと考えられる.
\begin{align}
\FP(t; S, T)
& =
\E_{P_S} [ B(S, T) | \mathcal{F}_t ]
= \E_{P_S} \Bigl[ \frac{B(S, T)}{B(S, S)} \Bigl| \mathcal{F}_t \Bigr] \nonumber \\
& =
\frac{B(t, T)}{B(t, S)} \nonumber
\end{align}

$F(t; S, T)$は$L(S, T)$のフォワードと考えられる.
\begin{align}
F(t; S, T)
& =
\E_{P_T} [ L(S, T) | \mathcal{F}_t ] \nonumber \\
& =
\E_{P_T} \Bigl[ \frac{1}{\tau} \Bigl( \frac{1}{B(S, T)} - 1 \Bigr) | \mathcal{F}_t \Bigr] \nonumber \\
& =
\E_{P_T} \Bigl[ \frac{1}{\tau} \Bigl( \frac{B(S, S)}{B(S, T)} - 1 \Bigr) | \mathcal{F}_t \Bigr] \nonumber \\
& =
\frac{1}{\tau} \Bigl( \frac{B(t, S)}{B(t, T)} - 1 \Bigr) \nonumber
\end{align}

\subsection{FRAs and interest rate swaps: the forward measure}
\begin{align}
V(t)
& =
B(t, T) \E_{P_T} [ \tau (K - L(S, T)) | \mathcal{F}_t ] = 0 \nonumber
\end{align}

\begin{align}
V(t)
& =
B(t, T) \tau K - B(t, T) \tau F(t; S, T) = 0 \nonumber
\end{align}

\begin{align}
\PS(t)
& =
\sum_{i=1}^n B(t, T_i) \E_{P_{T_i}} [ \tau_i (L(T_{i-1}, T_i) - K) | \mathcal{F}_t ] \nonumber \\
& =
\sum_{i=1}^n B(t, T_i) \tau_i (F(t; T_{i-1}, T_i) - K) \nonumber \\
& =
B(t, T_0) - B(t, T_n) - K \sum_{i=1}^n \tau_i B(t, T_i) \nonumber
\end{align}

\subsection{Option pricing in the forward measure}
\begin{align}
\BC(t; S, T, K)
& =
B(t) \E_Q \Bigl[ \frac{(B(S, T) - K)_+}{B(S)} \Bigl| \mathcal{F}_t \Bigr] \nonumber \\
& =
B(t) \E_Q \Bigl[ \frac{B(S, T)}{B(S)} 1_{\{ B(S, T) \geq K \}} \Bigl| \mathcal{F}_t \Bigr]
- K B(t) \E_Q \Bigl[ \frac{1}{B(S)} 1_{\{ B(S, T) \geq K \}} \Bigl| \mathcal{F}_t \Bigr] \nonumber \\
& =
B(t, T) P_T (B(S, T) \geq K | \mathcal{F}_t) - K B(t, S) P_S (B(S, T) \geq K | \mathcal{F}_t) \nonumber
\end{align}

\hrulefill
\begin{exercise}
\begin{align}
d \log \FP(t; S, T)
& =
\bigl( \Sigma(t, T) - \Sigma(t, S) \bigr) dW(t) - \frac{1}{2} \bigl( \Sigma(t, T)^2 - \Sigma(t, S)^2 \bigr) dt \nonumber
\end{align}
より
\begin{align}
\frac{d\FP(t; S, T)}{\FP(t; S, T)}
& =
\bigl( \Sigma(t, T) - \Sigma(t, S) \bigr) dW(t) + \Sigma(t, S) \bigl( \Sigma(t, S) - \Sigma(t, T) \bigr) dt \nonumber
\end{align}
\end{exercise}
\hrulefill \\

\begin{align}
\frac{dP_S}{dQ}
& =
\exp \Bigl( \int_0^S \Sigma(u, S) dW(u) - \frac{1}{2} \int_0^S \Sigma(u, S)^2 du \Bigr) \nonumber
\end{align}
であることとギルザノフの定理より
\begin{align}
W^S(t)
& =
W(t) - \int_0^t \Sigma(u, S) du \nonumber
\end{align}
は$P_S$の下でブラウン運動となる.
SDEに代入すると
\begin{align}
\frac{d\FP(t; S, T)}{\FP(t; S, T)}
& =
\bigl( \Sigma(t, T) - \Sigma(t, S) \bigr) dW^S(t) \nonumber
\end{align}
$\FP(S; S, T) = B(S, T)$より,このSDEを解くと
\begin{align}
B(S, T)
& =
\FP(t; S, T)
\exp \Bigl( \int_t^S \bigl( \Sigma(u, T) - \Sigma(u, S) \bigr) dW^S(u) - \frac{1}{2} \int_t^S \bigl( \Sigma(u, T) - \Sigma(u, S) \bigr)^2 du \Bigr) \nonumber
\end{align}

\begin{align}
v(t, S)
& =
\int_t^S \bigl( \Sigma(u, T) - \Sigma(u, S) \bigr)^2 du \nonumber
\end{align}
とおくと
\begin{align}
P_S (B(S, T) \geq K | \mathcal{F}_t)
& =
\mathrm{N} (d_-) \nonumber
\end{align}
ここで
\begin{align}
d_-
& =
\frac{1}{\sqrt{v(t, S)}} \Bigl( \log \frac{\FP(t; S, T)}{K} - \frac{1}{2} v(t, S) \Bigr) \nonumber \\
& =
\frac{1}{\sqrt{v(t, S)}} \Bigl( \log \frac{B(t, T)}{K B(t, S)} - \frac{1}{2} v(t, S) \Bigr) \nonumber
\end{align}

\begin{align}
W^T(t)
& =
W(t) - \int_0^t \Sigma(u, T) du \nonumber
\end{align}
は$P_T$の下でブラウン運動となる.
\begin{align}
W^S(t)
& =
W^T(t) + \int_0^t \bigl( \Sigma(u, T) - \Sigma(u, S) \bigr) du \nonumber
\end{align}
より
\begin{align}
B(S, T)
& =
\FP(t; S, T)
\exp \Bigl( \int_t^S \bigl( \Sigma(u, T) - \Sigma(u, S) \bigr) dW^T(u) + \frac{1}{2} \int_t^S \bigl( \Sigma(u, T) - \Sigma(u, S) \bigr)^2 du \Bigr) \nonumber
\end{align}
よって
\begin{align}
P_T (B(S, T) \geq K | \mathcal{F}_t)
& =
\mathrm{N} (d_+) \nonumber
\end{align}
ここで
\begin{align}
d_+
& =
\frac{1}{\sqrt{v(t, S)}} \Bigl( \log \frac{\FP(t; S, T)}{K} + \frac{1}{2} v(t, S) \Bigr) \nonumber \\
& =
\frac{1}{\sqrt{v(t, S)}} \Bigl( \log \frac{B(t, T)}{K B(t, S)} + \frac{1}{2} v(t, S) \Bigr) \nonumber
\end{align}

\hrulefill
\begin{exercise}
$X \sim \mathrm{N} (\mu, \sigma^2)$のとき
\begin{align}
P(x_0 e^X \geq K)
& =
\mathrm{N} \biggl( \frac{1}{\sigma} \Bigl( \log \frac{x_0}{K} + \mu \Bigr) \biggr) \nonumber
\end{align}
となる.
独立性の補題を用いて,結論を得る.
\end{exercise}
\hrulefill \\

\begin{theorem}
\begin{align}
\BC(t; S, T, K)
& =
B(t, T) \mathrm{N}(d_+) - K B(t, S) \mathrm{N}(d_-) \nonumber
\end{align}
\end{theorem}

\hrulefill
\begin{exercise}
\begin{align}
\BP(t; S, T, K)
& =
\BC(t; S, T, K) - B(t, T) + K B(t, S) \nonumber \\
& =
- B(t, T) \bigl( 1 - \mathrm{N}(d_+) \bigr) + K B(t, S) \bigl( 1 - \mathrm{N}(d_-) \bigr) \nonumber \\
& =
- B(t, T) \mathrm{N}(-d_+) + K B(t, S) \mathrm{N}(-d_-) \nonumber
\end{align}
\end{exercise}
\hrulefill \\

\end{document}
