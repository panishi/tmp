\documentclass[11pt,a4paper]{jsarticle}

\usepackage{amsmath,amssymb,amsthm}

\newtheorem{assumption}{Assumption}[section]
\newtheorem{proposition}[assumption]{Proposition}
\newtheorem{theorem}[assumption]{Theorem}
\newtheorem{exercise}{Exercise}[section]

\begin{document}

\appendix
\setcounter{section}{4}
\section{状態空間モデルにおける条件付き独立性に関する補足}
\subsection{(5.3) 式の導出}
\begin{flushleft}
\hspace{30pt} $p(x_t \mid x_{t+1}, y_{1:T})$
\end{flushleft}
\begin{flushleft}
\hspace{20pt} $= p(x_t \mid x_{t+1}, y_{1:t}, y_{t+1:T})$
\end{flushleft}
説明のしやすさを考慮して順番を入れ替えて
\begin{flushleft}
\hspace{20pt} $= p(x_t \mid y_{t+1:T}, x_{t+1}, y_{1:t})$
\end{flushleft}
$p(a, b \mid c) = p(a \mid b, c) p(b \mid c)$の関係より
\begin{flushleft}
\hspace{20pt} $= \displaystyle \frac{p(x_t, y_{t+1:T} \mid x_{t+1}, y_{1:t})}{p(y_{t+1:T} \mid x_{t+1}, y_{1:t})}$
\end{flushleft}
説明のしやすさを考慮して分子の中の順番を入れ替えて
\begin{flushleft}
\hspace{20pt} $= \displaystyle \frac{p(y_{t+1:T}, x_t \mid x_{t+1}, y_{1:t})}{p(y_{t+1:T} \mid x_{t+1}, y_{1:t})}$
\end{flushleft}
分子は$p(a, b \mid c) = p(a \mid b, c) p(b \mid c)$の関係より
\begin{flushleft}
\hspace{20pt} $= \displaystyle \frac{p(y_{t+1:T} \mid x_t, x_{t+1}, y_{1:t}) \, p(x_t \mid x_{t+1}, y_{1:t})}{p(y_{t+1:T} \mid x_{t+1}, y_{1:t})}$
\end{flushleft}
分子の1項目は条件付き独立性より
\begin{flushleft}
\hspace{20pt} $= \displaystyle \frac{p(y_{t+1:T} \mid x_{t+1}, y_{1:t}) \, p(x_t \mid x_{t+1}, y_{1:t})}{p(y_{t+1:T} \mid x_{t+1}, y_{1:t})}$
\end{flushleft}
\begin{flushleft}
\hspace{20pt} $= p(x_t \mid x_{t+1}, y_{1:t}).$
\end{flushleft}

ほぼ同じ議論により
\begin{align}
p(x_t \mid x_{t+1:T}, y_{1:T})
& =
p(x_t \mid x_{t+1}, y_{1:t}). \nonumber
\end{align}

\end{document}
